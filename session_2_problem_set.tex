\documentclass{article}
\usepackage[utf8]{inputenc}

\title{Math Mentorship Problem Set \#2 Solutions}
\author{Jack Ceroni}
\date{March 2019}

\begin{document}

\maketitle
\newpage
\section{Solutions}
\noindent
1. Let $T$ be some set in the plane. We wish to demonstrate that we can create arbitrarily small triangulations of this triangle, for the diameter of any of the triangulations must be less then $\epsilon$ for all $\epsilon \ > \ 0$. Let us consider the case where we mark points at the points representing half the length of each the sides of the triangle. We can then connect these three edges to create four smaller triangles within the larger triangle, for all triangles. If we call the set of the four smaller triangles (the triangulation) $t$, where a triangle in the triangulation is denoted by $t_i$, then we can assert that for the three triangles with one of their edges lying on the edge of the larger triangle, $t_i \ \sim \ T$ by $SAS$ (since the triangulation have side length that equivalent to the side lengths of the larger triangle, multiplied by $0.5$, and the angle lying between them is shared with the larger triangle). Through simple geometry, we can also prove that the triangle not lying on any of the edges in also similar (An easy method is through $AAA$, but there are other methods).
\newline\newline
For some arbitrary triangle, the diameter of that triangle is equal to sup$_{x, y \in T} \ d(x, \ y)$. This is equivalent to saying the diameter is equal to the longest side of the triangle. Since for all $t_i \ \in \ t \ \Rightarrow \ t_i \ \subset \ T$ (proper subset) for the method of triangulation that we have chosen, diam $t_i \ < \ $diam $T$. If we keep iterating on this process, first triangulating the triangle, then triangulating the triangulations, then triangulating that next iteration of triangulations and so on, in this same manner, and examine some arbitrary triangle (say, the top-most triangle in the set of triangulations), we can see that since the triangulations are similar to the larger triangle they are being triangulated from, that we are reducing the longest side of the the triangle by $\frac{1}{2}$ with each iteration.
\newline\newline
Finally, we can easily demonstrate that the series $\frac{D}{2^n}$ converges to $0$ (where $D$ the diameter of the initial triangle), implying that $\frac{D}{2^n} \ < \epsilon$, for any $\epsilon \ > \ 0$. To show convergence, we need to show that we can chose some $N$ with $n \ > \ N$, such that $|\frac{1}{2^n} \ - \ x| \ < \ \epsilon$. If $n \ > \ N$, then $2^n \ > \ 2^N$ and $2^n \ > \ N$, so $2^n \ > \ N$. Let $x \ = \ 0$ (since we are trying to prove convergence to $0$). We get $|\frac{D}{2^n}| \ < \ \epsilon$. Let $N \ > \ \frac{D}{\epsilon}$, which is always possible. We get $\frac{N}{D} \ > \ \frac{1}{\epsilon} \ \Rightarrow \ \frac{1}{\epsilon} \ > \ \frac{D}{N} \ > \ \frac{D}{2^n}$. We have demonstrated that this series converges to $0$ which as a direct part of the proof means that $\frac{D}{2^n}$ can always be less than $\epsilon$.
\newline\newline
2. The triangulated tetrahedron looks like the volume enclosed in tetrahedron subdivisions of a larger tetrahedron if we triangulate the surfaces of the larger tetrahedron, and then draw lines through the inside of the tetrahedron from the points on the surfaces, creating a set of subdivisions. The same proof as above holds for triangulations of tetrahedra, except we have to consider each of the surfaces of the tetrahedra triangulations in relation to the side of the larger tetrahedra that they are parallel to, and realize that \textbf{each} of the surfaces are similar to these larger surfaces, making the subdivisions "similar" to the larger tetrahedron. For higher dimensions simplices, triangulations are simply smaller subdivisions of the larger $n$-simplex.
\newline\newline
3. Simply triangulate the equilateral triangle as was outlined in the solution to question $1$. By the criteria for the Sperner labelling, for each vertex of the middle triangle (the one that has vertices lying on the midpoints of the sides of the larger triangle), all possible choices for our Sperner labelling of these vertices is the set of three-tuples ${(1, \ 2) \ \times \ (2, \ 3) \ \times \ (3, \ 1)}$. One of these tuples is in fact the set $(1, \ 2, \ 3)$ so we have found an appropriate Sperner labelling.
\newline\newline
4. Given an equilateral triangle, and triangulating by splitting it into smaller equilateral triangles by splitting the dies into equal fractions $\frac{1}{n}$ (for instance, $\frac{1}{2}$), we get triangulations of size $4$, $9$, $16$, etc. Take the lower right hand triangle. The lower right vertex will always be labelled $b$, the left vertex will be labelled either $b$ or $c$ and the top vertex will be labelled $a$ or $b$. Therefore, we can label the vertices $a$, $b$, $c$ (this demonstrates it doesn't matter in which order we label the vertices of the larger triangle).
\newline\newline
5. This can be generalized to tetrahedron by looking at the Sperner labelling of each of the the triangles making up the sides of the tetrahedron, and realizing that the same logic follows. The same logic also follows for high-dimensional simplices (by looking at the whole set of possible Sperner labellings).
\newline\newline
6. Let there be $2n$ triangulation vertices on the $(1, \ 2)$ side of a triangle with a Sperner labelling. It is apparent that the only place where the edge of a triangle will possibly have the labelling $(1, \ 2)$ on one of the edges of the larger triangle is on the edge of $T$ labelled $(1, \ 2)$. Let us pick an \textbf{optimal labelling} which is where every edge of the triangulation lying on $(1, \ 2)$ has the labelling of $(1, \ 2)$. For an even number of vertices (odd number of edges), we can simply alternate labellings on the vertices, such that every edge has a $(1, \ 2)$ labelling. Now, say we decide to make a switch, either $1 \ \rightarrow \ 2$ or $2 \ \rightarrow \ 1$. This will affect the $2$ edges attached to that vertex. From the optimal case, this will "negate" both edges on either side of the vertex, substracting $2$ from the number of triangles with $(1, \ 2)$ labellings. In future iterations, a switch could also negate only one edge, and add the other edge to the set of triangles with $(1, \ 2)$ labelling (net gain of $0$), or a switch could add both edges on either side of the vertex to the set (net gain of two). Now, adding or subtracting $2$ from an odd number of edges (we started off with an odd number of edges) will be an odd number, therefore this statement is true for an even number of vertices. 
\newline\newline
Now, let's look at the case with an odd number of vertices (even number of edges in total). We can \textbf{almost} create an optimal labelling with even vertices, however, there will always be a point where a vertex has both $1$ or both $2$ on either side, forcing us to be $1$ edge away from an optimal labelling. This mean we have an even number of edges to start off with, but we can initially negate one edge for our optimal case (maximum number of edges with $(1, 2)$ labelling). This is an \textbf{odd} initial number of edges. We can apply the same logic about "switches" from the even case, demonstrating that for an even number of vertices, only an odd number of $(1, \ 2)$ labellings is allowed. 
\newline\newline
7. Let one side of the triangle be labelled $(1, \ 2)$. The third vertex, $a$, must yield $(1, \ 2)$ labellings in the form $(a, \ 1)$ and $(a, \ 2)$, which is obviously impossible as $a$ would have to take on more than one value.
\newline\newline
8. Consider an element of the triangulation $t_i$ with an edge labelled $(1, \ 2)$ on the edge of the triangle. Let the path start at this point. Let the third vertex be labelled $a$. We can either label $a$ with $1$, $2$, or $3$. If we chose $1$ or $2$, the path can continue forward into the next triangle adjacent to this triangle. If we chose $3$, then the path ends. Say we decide to let the path continue. Since the triangle the path enters into next shares an edge with the previous triangle, it also has an edge $(1, \ 2)$, and a third vertex $b$. We again have the same choice of letting the path continue or stopping it. This continues for every triangle the path enters, until the path loops back around and exits through the triangle back at the $(1, 2)$ edge, or it stops in one of the $(1, \ 2, \ 3)$ triangles. Additionally, consider the triangulations of the $(1, \ 3)$ and $(2, \ 3)$ edges of the triangle. Since the path would have had to entered one of these triangles in some way, one of the edges must be $(1, \ 2)$. For the third vertex, instead of the choice between $1$, $2$, or $3$, we must chose between $1$ or $3$ and $2$ or $3$. Yet, it still simply follows that we can chose $1$ in the former case and $2$ in the latter to let the path continue, or chose $3$ to stop the path at this point.
\newline\newline
9. Let as take a collection of triangulations on the $(1, \ 2)$ edge, and let's say that we draw each of their respective paths into the triangle. Let us assume that there is no triangulation with the labelling $(1, \ 2, \ 3)$. This means that each path loops back around to the $(1, \ 2)$ edge and exits. Since there are an odd number of $(1, \ 2)$ edges on the boundary, there must exist two paths that start/exit through the same edge of the triangle. This means that if we think of the paths through the triangles that the path take as loops, where only one edge of the triangle is exposed to the "outside" of the loop through which one of the paths is taking (along with the fact that in order for a triangle in the triangulation to be a part of one of these paths, two of the edges must be labelled $(1, \ 2)$ for entry and exit. This means that the two loops have to "intersect" at some point, which would require a triangle to have all of its edges labelled $(1, \ 2)$, which we have already proven is impossible. Therefore, there must be some triangle labelled $(1, \ 2, \ 3)$ in the triangulation.
\newline\newline
10. For tetrahedra, we can generalize this proof by asserting that Sperner's lemma is true for each of the surfaces of the tetrahedron and the triangulation tetrahedra, so we simply have to pick a labelling such that a path passes through (or stops after passing through) faces of the triangulated tetrahedra. For high-dimensions, we simply have to consider the case of the dimension that is one dimension lower than it (For example, for tetrahedra, we know that Sperner's lemma is true for all of its spaces, and we adapt the theorem to work in 3-dimensions).
\newline\newline
11. Let $x_1, \ x_2, \ x_3 \ ...$ be some sequence. Let $x_n, \ x_{n+a}, \ x_{n+b}, \ ...$ with $a \ < \ b \ < \ c \ < \ ... $ be some subsequence of the sequence. If we chose some increasing set of numbers $A$, in the set of numbers indexing the subsequence. $B$, which is an increasing set of indices of the original sequence $C$, then it follows that $A$ is an increasing set of numbers in $C$, indexing the original sequence, therefore it follows that the subsequence of the subsequence is a subsequence of the original sequence.
\newline\newline
12. We need to show that there is some $N$ with $n \ > \ N$ such that $|x_n \ - \ x| \ < \ \epsilon$. By hypothesis, let us assert that this sequence does in fact converge to $1$. We get:
\newline
$$\Big | 1 \ - \ \displaystyle\frac{1}{n \ - \ 3} \ - \ 1 \Big | \ = \ \Big | - \displaystyle\frac{1}{n \ - \ 3} \Big | \ = \ \Big |\displaystyle\frac{1}{n \ - \ 3} \Big | \ < \ \epsilon$$
\newline
We know that $n \ > \ N$. Let use chose $N \ > \ \frac{1}{\epsilon} \ + \ 3$. It follows that $n \ > \ \frac{1}{\epsilon} \ + \ 3 \ \Rightarrow \ n \ - \ 3 \ > \ \frac{1}{\epsilon} \ \Rightarrow \ \frac{1}{n \ - \ 3} \ < \ \epsilon$. Therefore, we have shown that we can chose some appropriate $N$ to make the definition for convergence of a sequence hold true.
\newline\newline
13. Conceptually, this statement makes sense. If we continue to subdivide a triangle and pick points that are strictly increasing in a sequence, as the triangles get smaller and smaller, they will eventually converge towards one point, which is representative of $y$ within the sequence. In addition, whenever we are triangulating and picking a new triangle, we are creating subsequences of the previous subsequence, which are all subsequences of the original sequence. As the triangles get smaller and smaller, we approach a subsequence of the original sequence in the form of $a, \ a, \ a, \ …$, which is convergent. For higher dimensional simplices, we must pick points contained with the region bounded by the surfaces of the complex, and continue triangulating by breaking the complex down into smaller and smaller subdivisions. The same logic as the the two dimensional case then holds.
\end{document}
